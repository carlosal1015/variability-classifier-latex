\begin{minipage}{\linewidth}

\selectlanguage{english}
\begin{abstract}
This work assesses the automated classification of periodic variables based on their photometric variability. We extract a variety of features from $\approx 30000$ lightcurves in the EROS data set, including the period of the signal using the Lomb--Scargle algorithm, as well as conditional entropy (CE). In this context, we present a runtime--optimized, fast Python/Cython implementation of the CE algorithm. To provide further separation of quasars (QSOs) in feature space, we implement the structure function as a measure for the intrinsic stochastic variability of the source. Based on these features, we train a Support Vector Machine (SVM) and a Random Forest classifier for supervised classification into classes and subclasses of all occurring variables, optimize the hyperparameters for both models using grid--search in the parameter space to achieve optimal classification performance. We perform a $5$--fold cross--validation and achieve $\cdots$ with the SVM, and $\cdots$ with the Random forest classifier for the main catgories, and $\cdots$ for the sub--categories.
\end{abstract}

\selectlanguage{german} 

\begin{abstract}
Im Rahmen dieser Arbeit $\cdots$
\end{abstract}

\end{minipage}

\selectlanguage{english}

% Name the different variables in the EROS dataset
% Name precise number of features
% Name different color bands