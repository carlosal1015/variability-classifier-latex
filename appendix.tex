\begin{appendices}
\chapter{Python/Cython implementation of the CE algorithm}
 
\begin{minted}[fontsize=\scriptsize]{python}

#cython: boundscheck=False
#cython: wraparound=False

import numpy as np
cimport numpy as np
cimport cython

cdef fasthistogram2d(np.ndarray data, np.ndarray edges):

	x, y = np.digitize(data[:,0], edges[1:-1]), np.digitize(data[:,1], edges[1:-1])
	nbins = len(edges) - 1
	flatcount = np.bincount(x + y * nbins, minlength = nbins * nbins)
	return flatcount.reshape((nbins, nbins)).T

def ce(double period, np.ndarray t, np.ndarray m, unsigned int nbins = 10):

	''' Returns the conditional entropy of the phase-folded signal. '''

	cdef np.ndarray phase, phasedt
	cdef np.ndarray histogram, edges
	cdef np.ndarray normalized, positive, p
	cdef double mmin, mmax, pmin, pmax

	assert len(t) == len(m)
	if len(t) == 0 or period <= 0:
		return np.inf

	# Phase-fold the lightcurve
	phase = lambda t, period: np.divide(np.mod(t, period), period)
	phasedt = phase(t, period)

	# Rescale magnitude to be between 0.0 and 1.0
	# We want the data to occupy a unit square in the phase-magnitude plane
	mmin, mmax = np.min(m), np.max(m)
	m = (m - mmin) / (mmax - mmin)

	# Rescale phasedt to be between 0.0 and 1.0
	phasedt = (phasedt - np.min(phasedt))
	pmin, pmax = np.min(phasedt), np.max(phasedt)
	phasedt = (phasedt - pmin) / (pmax - pmin)

	# Bin the data in two dimensions
	edges = np.linspace(0, 1, nbins + 1)
	histogram = fasthistogram2d(np.vstack((phasedt, m)).T, edges)
	# Normalizing the histogram leaves us with the occupation probability for every bin
	normalized = histogram / float(len(phasedt))
	positive = (normalized > 0)
	# This will contain the sums of each column
	p = nsum(normalized, axis = 0)
	# Repeat the array row-wise, so that it can be sliced by mask
	p = np.repeat(np.reshape(p, (1,-1)), nbins, axis = 0)
	return nsum(normalized[positive] * np.log(p[positive] / normalized[positive]))
  
\end{minted}

\end{appendices}