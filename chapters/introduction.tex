% All about "turning data into knowledege"
% Predicting the future based on the past
% For a very long time, people thought of stars to be invariable and eternal...
% Data flood in astronomy
% Hipparcos and Gaia

% Criteria for classifiers: Accuracy/Performance, Computational complexity, Scalability, Interpretability of the model, Simplicity of the model, Preprocessing work

% In the spirit of reproducible research (Donoho et al. 2008) all of the code and data necessary to reproduce all of the figures in this paper are included as auxiliary material.

We will cover the theoretical background in section \ref{sec:theory}, from a brief description of important variable objects (\ref{sec:theory-variable-objects}) and the underlying mechanisms for different types of photometric variability (\ref{subsec:intrinsic-extrinsic-variability}), to the characteristics of the respective light curves (\ref{subsec:light-curves-variables}). Section \ref{sec:statistical-analysis-time-series} explains two different algorithms for period--finding (\ref{subsec:period-finding}) and introduces the structure function as a way to characterize stochastic variability (\ref{subsec:structure-function}). We then move on to a short introduction to the basic concepts of Machine Learning (\ref{sec:introduction-machine-learning}) and outline three different algorithms for classification (\ref{subsec:algorithms-classification-and-regression}). The main part of the thesis, section \ref{sec:main}, details all the steps undertaken for the classification of variables in the LMC, from the feature extraction (\ref{sec:feature-extraction}), to the training process and hyperparameter optimization for all different classifiers for both superclasses and subclasses. A description of the training data set can be found in section (\ref{sec:about-training-set}). We provide a discussion of the results in section \ref{sec:discussion}, amongst other things comparing the performance of the individual classifiers, the significance of the features we used, and conclude with suggestions for further improvement in future work.