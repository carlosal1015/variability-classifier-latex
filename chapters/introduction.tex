% All about "turning data into knowledege
% Write about what Machine Learning has brought us
% Predicting the future based on the past

% In the spirit of reproducible research (Donoho et al. 2008) all of the code and data necessary to reproduce all of the figures in this paper are included as auxiliary material.

Being an observational science without the direct possibility of physical interaction with the objects of interest, astronomy is usually limited to infer information from the light received by detectors. This involves analysing the distribution of the light in wavelength, energy, polarization, position on the sky or \emph{time}. This work focuses on the latter, specifically in the context of \emph{variable objects} or \emph{variables}. Variables change their brightness over time, and this variability directly enables us to test models of given stars by comparing the observed light curves to the model's predictions. A systematic observation of the variability characteristics can therefore provide valuable insight on cosmic evolution and stellar properties, such as temperature, mass and radius \citep{percy2007}. Pulsating variables, \eg Cepheids and RR Lyrae, have been invaluable to calibrate the cosmological distance ladder because of their well--established period--luminosity relationship.\\

As technology has evolved over the decades, astronomers have gone from photographic plates to modern photometry surveys, producing massive amounts of data. Since humans are neither willing nor able to process huge time--series, this data deluge requires automated approaches to process the data in a way that leads to scientific insights. Supervised Machine Learning algorithms are powerful statistical models to deal with this kind of problems, but their performance in terms of accuracy and computational complexity has to be carefully assessed for the specific problem to be solved.\\

We will cover the theoretical background in section \ref{sec:theory}, from a brief description of important variable objects (\ref{sec:theory-variable-objects}) and the underlying mechanisms for different types of photometric variability (\ref{subsec:intrinsic-extrinsic-variability}), to the characteristics of the respective light curves (\ref{subsec:light-curves-variables}). Section \ref{sec:statistical-analysis-time-series} explains two different algorithms for period--finding (\ref{subsec:period-finding}) and introduces the structure function as a way to characterize stochastic variability (\ref{subsec:structure-function}). We then move on to a short introduction to the basic concepts of Machine Learning (\ref{sec:introduction-machine-learning}) and outline three different algorithms for classification (\ref{subsec:algorithms-classification-and-regression}). The main part of the thesis, section \ref{sec:main}, details all the steps undertaken for the classification of variables in the LMC, from the feature extraction (\ref{sec:feature-extraction}), to the training process and hyperparameter optimization for all different classifiers for both superclasses and subclasses. A description of the training data set can be found in section (\ref{sec:about-training-set}). We provide a discussion of the results in section \ref{sec:discussion}, amongst other things comparing the performance of the individual classifiers, the significance of the features we used, and conclude with suggestions for further improvement in future work.