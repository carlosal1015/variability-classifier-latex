\section{About the EROS training set}

We use a training data set provided by \citet{kim2014} which contains labels (main and sub--classes) and lightcurves for $32655$ sources in the Large Magellanic Cloud (LMC). The lightcurves were recorded by the Expérience pour la Recherche d’Objets Sombres (EROS) project, a wide--field survey. A thorough description of the compilation process is available in \citet{kim2014}, however, here is a list of the main steps:

\begin{enumerate}
\item Compile known periodoic variables in the LMC from the OGLE and MACHO surveys
\item Add $982$ blue variables (BVs) from the MACHO database
\item Add 565 quasi--stellar objects (QSOs)
\item 
\end{enumerate}

% Number of sources in the data set
% Not a standard astronomical B and R bands

\section{Feature Extraction \& Feature Selection}

We extract a variety of features characterizing the variability of the source. A lot of them are standard statistical features, but some are more sophisticated, trying to incorporate a model of the underlying physics. In the following, $N$ will be the number of data points in the light curve, $(t_i, m_i)$ the $i^{\text{th}}$ data point in the light curve.

\begin{enumerate}
%\setlength{\itemsep}{5pt}
%\setlength{\parskip}{0pt}
%\setlength{\parsep}{0pt}
%\setlength{\abovedisplayskip}{.5em}
%\setlength{\belowdisplayskip}{.5em}

\litem{$\mu$ (Mean of magnitude)} The arithmetic mean of the magnitude, given by
\begin{equation}\mu = \frac{1}{N} \sum\limits_{i=1}^{N} m_i.\end{equation}

\litem{$\sigma$ (Standard deviation of magnitude)} The standard deviation of the magnitude, given by
\begin{equation}\sigma = \sqrt{\frac{1}{N} \sum_{i=1}^N (m_i - \mu)^2}.\end{equation}

\litem{$Q_{50}$ (Median of magnitude)} The median of the magnitude, given by
\begin{equation}Q_{50} = \cdots.\end{equation}

\litem{$\bar \mu$ (Weighted mean of magnitude)} The weighted arithmetic mean of the magnitude, given by
\begin{equation}\bar \mu = \big(\sum\limits_{i=1}^{N} w_i m_i\big) \; / \; \big(\sum\limits_{i=1}^{N} w_i\big).\end{equation}
% @TODO: Specify weights

\litem{$\bar \sigma$ (Weighted standard deviation of magnitude)} The weighted standard deviation of the magnitude, given by
\begin{equation}\bar \sigma = \cdots.\end{equation}
% @TODO: Specify weights
% @TODO: Add formula for weighted standard deviation

\litem{$\gamma_1$ (Skewness)} The skewness of the magnitude, given by
\begin{equation}\gamma_1 = \cdots.\end{equation}

\litem{$\gamma_2$ (Kurtosis)} The kurtosis of the magnitude, given by
\begin{equation}\gamma_2 = \cdots.\end{equation}

\litem{$Q_{25}$ (25\% quartile)} The 25\% quartile of the magnitude, given by
\begin{equation}Q_{25} = \cdots.\end{equation}

\litem{$Q_{75}$ (75\% quartile)} The 75\% quartile of the magnitude, given by
\begin{equation}Q_{75} = \cdots.\end{equation}

\litem{$\text{IQR}$ (Interquartile range)} The interquartile range of the magnitude, given by
\begin{equation}\text{IQR} = Q_{75} -Q_{25}.\end{equation}

\litem{$P_{\text{LS}}$ (Lomb--Scargle period)} The period of the signal according to the Lomb--Scargle algorithm, given by
\begin{equation}P_{\text{LS}} = \frac{1}{2 \sigma_y^2} \Bigg\{ \frac{\big[\sum\limits_{i=1}^k (y_i - \mu_y) \cos(\omega(t_i - \tau))\big]^2}{\sum\limits_{i=1}^k \cos^2(\omega(t_i - \tau))} + \frac{\big[\sum\limits_{i=1}^k (y_i - \mu_y) \sin(\omega(t_i - \tau))\big]^2}{\sum\limits_{i=1}^k \sin^2(\omega(t_i - \tau))}\Bigg\}\end{equation}

\litem{$\text{FAP}_{\text{LS}}$ (False--Alarm probability for Lomb--Scargle)} The false--alarm probability (FAP) for the Lomb--Scargle algorithm, given by
\begin{equation}\text{FAP}_{\text{LS}}(x) = 1 - (1 - \euler^{-x})^M.\end{equation}

\litem{$\text{SNR}$ (Signal--to--noise ratio)} The signal--to--noise ratio, given by
\begin{equation}\text{SNR} = \frac{\mu}{\sigma}.\end{equation}

\litem{$S$ (Shannon entropy)} The Shannon entropy of the signal, given by
\begin{equation}S = -\sum\limits_{i=1}^N m_i \ln(m_i).\end{equation}

\litem{$\eta$ ($\cdots$)} The $\eta$ feature as proposed by \citep{}, given by
\begin{equation}\eta = \frac{1}{\sigma^2 (N-1)} \sum\limits_{i=1}^{N-1} (m_{i+1} - m_{i})^2.\end{equation}

\litem{$\cdots$ (Half--magnitude--amplitude ratio)} The ratio between higher \resp lower amplitudes than average, given by
\begin{equation}\cdots = \cdots.\end{equation}

\litem{$\text{MAD}^{(1)}$ (Mean absolute deviation)} The mean absolute deviation of the magnitude, given by
\begin{equation}\text{MAD}^{(1)} = \cdots.\end{equation}

\litem{$\text{MAD}^{(2)}$ (Median absolute deviation)} The median absolute deviation of the magnitude, given by
\begin{equation}\text{MAD}^{(2)} = \cdots.\end{equation}

\litem{$\text{SF}_A$ (Structure function $A$)} The ...
\begin{equation}\text{SF}_A = \cdots.\end{equation}

\litem{$\text{SF}_\gamma$ (Structure function $\gamma$)} The ...
\begin{equation}\text{SF}_\gamma = \cdots.\end{equation}

\litem{$\mathcal{F}$ (Fourier series)} We fit standard fourier series with five terms to the phase--folded lightcurve.
\begin{equation}\mathcal{F}(t) = \frac{A_0}{2} + \sum_{k=1}^{\infty} ( A_k \cos(2 \pi k t) + B_k \sin(2 \pi k t) ).\end{equation}

% CE – three candidate periods + scores
% sf-A-error, sf-gamma-error
% Slope percentile features
% fourier-A-i, fourier-phi-i
% fourier-residuals

% Shapiro-W und Shapiro-p
% Proper citing of all equations

\end{enumerate}

This leads to a total of $\cdots$ features in $\cdots$ bands.

% State that feature are highly correlated
% Show histogram of some features.
% Show some scatterplots
% More about period finding

\section{Hyperparameter Optimization}

% Grid search
% Show table with the optimal results

\section{Performance of the Support Vector Machine (SVM)}

% Hyperparameter optimization
% Add confusion matrix for main classes
% Add confusion matrix for subclasses

\renewcommand{\arraystretch}{1.5}
\resizebox{\textwidth}{!}{
\begin{tabular}{c|ccccccccc|c}
\toprule
%& \multicolumn{8 }{c}{Predicted class} & & \\
%\hline
                 & BV        & CEPH       & DSCT      & EB         & LPV         & NoneVar    & QSO       & RRL        & T2CEPH   & Recall ($\%$) \\
\hline
BV               & {\bf 759} &            &           &            &             &            &      2    &            &          &        \\
CEPH             &     2     & {\bf 2201} &           &            &             &            &           &            &          &        \\
DSCT             &           &       1    & {\bf 421} &            &             &            &           &            &          &        \\
EB               &    16     &      19    &           & {\bf 3290} &             &            &      4    &            &          &        \\
LPV              &     4     &      11    &           &            & {\bf 15878} &            &      2    &            &          &        \\
NoneVar          &    18     &       2    &           &            &             & {\bf 4607} &      12   &            &          &        \\
QSO              &     3     &      25    &           &            &             &            & {\bf 142} &            &          &        \\
RRL              &           &      35    &           &            &             &            &      1    & {\bf 4349} &          &        \\
T2CEPH           &           &            &           &            &             &            &           &            & {\bf 1}  &        \\
\bottomrule
Precision ($\%$) &           &            &           &            &             &            &           &            &          &        \\
\end{tabular}
}

\section{Performance of the Random Forest Classifier}

% Hyperparameter optimization
% Add confusion matrix for main classes
% Add confusion matrix for subclasses
% Add feature importance for both

\chapter{Assessing Classification Performance for GAIA}

% Simulating GAIA time series
% Performance of the best classifier on GAIA data