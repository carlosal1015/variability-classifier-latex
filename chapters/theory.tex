\section{Variable Objects in Astronomy}
\label{sec:theory-variable-objects}

% For a very long time, people thought of stars to be invariable and eternal...

When observing the sky over a sufficient time span, we empirically find that a lot of objects are subject to flux variability, \ie their observed magnitude $m$ is a function of time, even on short time--scales\footnote{Evidently, on the time span of their lifetime, all stars show some kind of variability as they evolve. However, this is not what we mean by the term \emph{variability}.}. These significant changes in brightness can be somewhere between some thousandths of its magnitude and multiples of a magnitude. A systematic observation of variables can provide valuable insight on cosmic evolution and stellar properties, such as temperature, mass and radius \citep{percy2007}. Some variables change very rapidly with periods of several hours, while others take years to complete one cycle. Some are strictly periodic, while others show  rather stochastic, non--deterministic behaviour, which is much harder to characterise. Ultimately, it all boils down to the reason for variability, which gives rise to different a categories of variables. There are numerous types of variable objects with all kind of different mechanisms for variability.

% Can not explain all different classes in this work

\subsection{Intrinsic \& Extrinsic Variability}

We say that an object shows \emph{intrinsic} variability, when the change in brightness is caused by real changes in the physical state of the object. Sources with intrinsic variability can be roughly grouped into \emph{pulsating}, \emph{eruptive} and \emph{catalysmic} or \emph{explosive} variables.

\begin{itemize}
\item Pulsating variables are oscillating in radius $R$ because they are not in hydrostatic equilibrium. The gravitational field drives contraction, which , while the nuclear fusion $\cdots$. 
% kappa mechanism, epsilon mechanism
	\begin{itemize}[label=$\circ$]
	\item Classical Cepheids or $\delta$--Cepheids are bright stars with masses of $5$--$10 \, \unit{M_\odot}$, mostly in the helium--core--burning phase, that have periods between $1.0$--$50.0 \, \unit{d}$ \citep{cox1980}, and can be found in an almost vertical, narrow strip in the Hertzsprung--Russell diagram known as the \emph{instability strip}. They belong certainly to the most important variables, because of their well established \emph{period--luminosity} relation
	\begin{equation}
	\label{eq:period-luminosity-relation}
	\cdots = \cdots,
	\end{equation}
	which allows astronomers to use Cepheids as \emph{standard candles} for distance estimation. Standard candles are objects with a known absolute magnitude $M$, which can then be used to calculate the distance modulus $\mu = m - M$ by measuring the apparent magnitude. The object's distance is then given by
	\begin{equation}
	\log_{10}(d) = 1 + \frac{\mu}{5},
	\end{equation}
	assuming negligible interstellar absorption \citep{hanslmeier2007}. 
	% Period-luminosity relationship	
	% Rapid increase, rather slow decrease
	% Periods between 1 and 50 days
	% Bright
	\item $\delta$-Scuti stars or \emph{Dwarf Cepheids} are short--periodic ($0.7$-$5.0 \, \unit{h}$) variables with small amplitudes and typical masses of $1$-$2 \, \unit{M_\odot}$.
	% Dwarf cepheids
	\item RR Lyrae are somewhat smaller, fainter objects than Cepheids. They have masses between $0.5$--$0.6 \, \unit{M_\odot}$ and typical periods between $0.2 $--$1.2 \, \unit{d}$ \citep{unsoeld2001}. Similar to Cepheids, RR Lyrae can be used as standard candles, although this could be harder for far away objects due to the fainter nature.
	% Also used as distance estimators
	% Smaller than Cepheids, shorter periods, lower luminosity
	\item Long--periodic variables
	\end{itemize}
\item Eruptive variables $\cdots$
\item Catalysmic \& Explosive variables $\cdots$
\end{itemize}

On the other hand, if the change in brightness as perceived from the earth is merely due to external influences, a variable exhibits \emph{extrinsic} variability.

\begin{itemize}
\item Rotating variables are $\cdots$
\item Eclipsing binaries are $\cdots$
% Majority of stars in milkway are binary systems
\item Planetary transits are $\cdots$
\end{itemize}

\subsection{Stochastic variability \& QSOs}
 
We want to pay special attention to stochastic variability...
 
% Pulsating variables 
% Cepheids 
% $\delta$--Scutis
% RR Lyrae
% Eclipsing binaries
% Active stars, flares and spots

\subsection{Lightcurves of Variable Stars}

% Cepheids: rapid rise to maximum, then slow decrease

\section{Statistical Analysis of Time Series}

When observing one specific object over some time interval $\Delta t$, we can record its magnitude $m_i$ at time $t_i$ and obtain a so-called \emph{time series}. A time series $\Tau$ is a sequence of $N$ data points $(t_i, \vec y_i),\; t_i,(y_i)_j \in \mathbb{R}$ with $t_i < t_{i+1} \; \forall i = 1,\ldots,N$. We call $\Tau$ \emph{evenly--sampled} $\Leftrightarrow t_{i+1} - t_i = C \in \mathbb{R}$ where $C$ is constant. Otherwise, $\Tau$ is \emph{unevenly--sampled}.\\

In section \ref{sec:theory-variable-objects}, we have seen that there are a whole variety of different mechanisms for variability. These mechanisms will lead to distinct lightcurves, giving rise to discriminative features that can later be used for classification.

\subsection{Algorithms for Period-finding}

One very important feature of a periodic signal is, of course, its period $T$. However, reconstructing $T$ from the time series $\Tau$ is not trivial, especially if the data is very noisy. Ultimately, this is a regression problem itself, see subsection \ref{sec:introduction-machine-learning}, however, we will not tackle this with Machine Learning, but rather make use of some well--known methods devised for this very problem. The ultimate objective is to find the \emph{periodogram} $P(\omega)$ of signal, which is an estimator for the \emph{spectral density} $P$ as a function of the angular frequency $\omega = \frac{2 \pi}{T}$\footnote{In this regard, the periodogram can be seen as spectrum (some quantity over frequency) of a time series $\Tau$.}. Traditionally, there are a variety of methods available to characterize the signal in frequency--domain, mostly Fourier analysis. In 1898, Arthur Schuster defined the \emph{Schuster periodogram}

\begin{equation}
\label{eg:schuster-periodogram}
P_{\text{S}}(\omega) = \frac{1}{k} \, \biggl| \, \sum\limits_{i=1}^{k} y_i \euler^{-\mathrm{i} \omega t_i} \, \biggr|^2,
\end{equation}

% Mention DFT and FFT, N*log(N)
% If omega0 is true period than e^{-i * omega0 * t_i} will have large contribution
% https://charlesmartin14.wordpress.com/2012/11/05/noisy-time-series/
% Reference for Schuster periodogram

% @TODO: Check again about the |F(omega)|
% @TODO: Add a reference for "because they tend to boost long--periodic noise"

for a time series $(t_i, y_i)$ of length $k$. Note that equation \eqref{eg:schuster-periodogram} is essentially given by $\| \mathcal{F}(\omega) \|_2 $, the $L^2$ norm of the discrete Fourier transform (DFT) $\mathcal{F}$ of the signal. Therefore, $P_{\text{S}}(\omega)$ can be approximated very efficiently when using the fast Fourier transform algorithm (FFT) \citep{cooley1965}, which allows for a runtime complexity of $\mathcal{O}(N\log{(N)})$ instead of $\mathcal{O}(N^2)$. However, it can be shown that those Fourier transform--based methods perform poorly on unevenly--sampled data, because they tend to boost long--periodic noise. Since evenly--sampled data can be difficult to obtain, especially for earth--bound surveys in astronomy, a lot of effort was put into developing suitable methods that can cope with this kind of data. This work makes use of two popular methods devised for unevenly--sampled data, one based on least--squares spectral analysis (LSSA), namely the \emph{Lomb--Scargle periodogram}, and the other one based on the minimization of some metric for dispersion in phase space, in this case \emph{conditional entropy}. We provide a brief overview for both of them.

\subsubsection{Lomb--Scargle Periodogram}

The most prominent method for finding the period of an unevenly--sampled lightcurve was developed by \citet{lomb1976} and extended by \citet{scargle1982}. It is devised to find the period of a sinusoidal-shaped periodic signal. For a given time--series $(t_i, (y_i, \sigma_i))$ of length $k$ with arithmetic mean $\mu_y$ and variance of the noise $\sigma_y^2$, Scargle defines the \emph{normalized Lomb--Scargle periodogram} as

\begin{equation}
\label{eq:normalized-lomb-scargle}
P^{\text{N}}_{\text{LS}}(\omega) = \frac{1}{2 \sigma_y^2} \Bigg\{ \frac{\big[\sum\limits_{i=1}^k (y_i - \mu_y) \cos(\omega(t_i - \tau))\big]^2}{\sum\limits_{i=1}^k \cos^2(\omega(t_i - \tau))} + \frac{\big[\sum\limits_{i=1}^k (y_i - \mu_y) \sin(\omega(t_i - \tau))\big]^2}{\sum\limits_{i=1}^k \sin^2(\omega(t_i - \tau))}\Bigg\},
\end{equation}

where $\tau$ is a time--offset defined by

\begin{equation}
\tan(2 \omega \tau) = \frac{\sum\limits_{i=1}^k \sin(2 \omega t_i)}{\sum\limits_{i=1}^k \cos(2 \omega t_i)},
\end{equation}

which makes the periodogram $P^{\text{N}}_{\text{LS}}(\omega)$ invariant for a translation in $t$. Furthermore, this time--offset $\tau$ makes the periodogram identical to least--squares fitting of a single--component sinusoidal model of the form $y(t) = A \sin(\omega t + \varphi)$ \citep{horne1986, vanderplas2015}. Additionally it can be shown that --- if the periodogram is properly normalized\footnote{There has been a lively discussion about the correct normalization of the periodogram, which retains above convenient statistical properties, see \citet{lomb1976,astroML,zechmeister2009}. The correct normalization factor in equation \eqref{eq:normalized-lomb-scargle} is \emph{the total variance of the data}, as explained in detail in \citet{horne1986}.} --- $P^{\text{N}}_{\text{LS}}(\omega_0) \propto \exp(-x)$ for pure noise at any frequency $\omega_0$, where $x$ denotes the height of the peak. This exponential probability distribution gives rise to the \emph{false--alarm-probability (FAP)},

\begin{equation}
\text{FAP}_{\text{LS}}(x) = 1 - (1 - \euler^{-x})^M,
\end{equation}

which is a suitable and convenient estimator for the significance\footnote{In this case: The probability that the peak is the product of a true signal, rather than the result of randomly distributed noise.} of a peak, and can be identified as the probability that a peak of at least height $x$ will occur in a periodogram on a grid of $M$ independent frequencies \emph{when evaluated on pure noise} \citep{horne1986}.\\

In spite of its usefulness, there are some drawbacks of the normalized Lomb--Scargle periodogram in its original formulation:

\begin{itemize}
\item $P^{\text{N}}_{\text{LS}}(\omega)$ given by equation \eqref{eq:normalized-lomb-scargle} does not account for measurement errors, \eg by introducing weights for all data points. This however ignores an important aspect of the measurement process, where some data points might in fact be more accurate than others, for example due to changing conditions in the measurement environment.
\item Offset $\cdots$ % @TODO: Elaborate on constant offset
\end{itemize}

% This led to further generalization...
% The \emph{generalized Lomb--Scargle periodogram} is less prone to aliasing, and provides more accurate results (http://arxiv.org/abs/0901.2573).

% Generalized and classical
% Bayesian generalized Lomb-Scargle (Martiniee)
% Single point estimate rather than a posterior probability distribution
% Multi-band periodogram and matrix formalism
% It is possible to carefully correct for such aliasing by iteratively removing contributions from the estimated window function (e.g. Roberts et al.1987), we’ll ignore this detail in the current work.
% Discrete and continous, Fast Fourier Transformation (FFT)
% Remarks concerning runtime N^2 and Nlog(N) implementations

\subsubsection{Conditional Entropy}

Using conditional entropy for period finding was first proposed by \citet{graham2013} as an extension of a similar algorithm developed by \citet{cincotta1995}. Both algorithms look at the problem from an information theory point of view. The fundamental idea is that the phase-folded lightcurve will form the most ordered arrangement of points in phase space when folded with the true period, whereas ...

\begin{equation}
H_S = - \sum_{i=1}^k \mu_i \ln(\mu_i)
\end{equation}
\begin{equation}
H_c = \sum_{i,j} p(m_i, \phi_i) \ln(\frac{p(\phi_j)}{p(m_i, \phi_j)})
\end{equation}

% Works good for evenly sampled datapoints

%We provide a runtime--optimized, fast \emph{Cython} implementation of the conditional entropy period--finding algorithm.

\subsection{Structure Function}

\section{Statistical Inference using Machine Learning}

\subsection{Introduction to Machine Learning}
\label{sec:introduction-machine-learning}

Machine Learning (ML), sometimes referred to as \emph{pattern recognition}, is a subfield of Artificial Intelligence (AI) primarily dealing with automated approaches to infer meaning from data in such a way that, after some training process, a model is obtained, that can later be used to make predictions or decisions on similar data or achieve some other kind of insight. This is a form of learning, because rather than explicitly programming certain decision rules, we want to generalize observations towards an underlying model that performs as good as possible when confronted with new data.\\

% @TODO: Elaborate on reinforcement learning
Depending on the problem to be solved and the data available, machine learning algorithms can be divided into \emph{supervised}, \emph{unsupervised}, \emph{reinforcement} learning and some hybrid form of these. In supervised learning, we train our model by providing examples of the desired output, whereas in unsupervised learning, $\cdots$. Reinforcement learning works by $\cdots$.\\

There are a variety of learning problems, such as \emph{classification problems}, \emph{regression problems} and \emph{clustering}, but also \emph{dimensionality reduction} and \emph{density estimation} are subject to Machine Learning.\\

% Explain classification, regression and clustering
% Describe supervised, unsupervised and reinforcment learning
% Large amounts of data, computationally efficient
% Elaborate on why machine learning vs. directly programming decision rules
% Also to given insight: Sometimes humans can not really describe how they are doing things
% Overfitting, underfitting
% Add some examples for machine learning problems

Please note that in practise both $N$ and $d$ can be very large, and inferring $\varphi$ from $\trainingdata$ directly can become computationally expensive or even unfeasible.

\subsubsection{Classification \& Regression}

In this work, we are primarily interested in classification and regression using supervised methods. Therefore, we are typically trying to solve the following problem: Given some training data

\begin{equation}
\trainingdata = \{ (\vec x_i, y_i) \where \vec x_i \in \mathbb{R}^d, y_i \in \classes \quad \forall i = 1,\ldots,N \},
\end{equation}

consisting of $N$ \emph{feature} vectors $\vec x_i \in \observations$ of length $d$ and $N$ \emph{labels} or \emph{classes} $y_i \in \classes$, find some mapping function $\varphi \colon \observations \to \classes$ such that

\begin{equation}
\estimator(\vec x_i) = y_i \in \classes \quad \forall i = 1,\ldots,N,
\end{equation}

where $y_i$ denotes the true class of an arbitrary observation $\vec x$ out of all possible observations $\observations$. In any case, $\varphi$ is an \emph{estimator}. If the label $y_i$ is a discrete value, $\varphi$ will be called \emph{classifier}, if it is continuous, $\varphi$ will be called \emph{regressor}.\footnote{The term \emph{probabilistic classifier} stands for an estimator that assigns a probability $p_{\text{class}}$ for every class to the input vector. Based on this, classification can be conducted, \eg by maximum likelihood.} We consider $\varphi$ to be optimal if, confronted with some testing data,

\begin{equation}
\testingdata = \{ (\vec x'_j) \where \vec x'_j \in \mathbb{R}^d \quad \forall j = 1,\ldots,N \},
\end{equation}

the estimator yields

\begin{equation}
\hat y_j := \varphi({\vec x'_j}) = y_j \quad \forall j = 1,\ldots,N,
\end{equation}

where $\hat y_j$ is the estimators response, and $y_j$ denotes the true class of $\vec x'_j$.

\subsubsection{Evaluating Models}

% Write about generalization, overfitting and underfitting

The estimators performance can be assessed by a variety of metrics: For classification, we define accuracy $A$, precision $P$ and recall $R$ as

\begin{equation}
A = \frac{T_p + T_n}{T_p + F_p + T_n + F_n}, \; P = \frac{T_p}{T_p + F_p}, \; R = \frac{T_p}{T_p + F_n},
\end{equation}

% @TODO: Elaborate on model evaluation

where $T_p, F_p, T_n, F_n$ denote the number of \emph{true positives}, \emph{false positives}, \emph{true negatives} and \emph{false negatives}\footnote{Accuracy, precision and recall are sometimes also referred to as \emph{}, \emph{}, \emph{}}. For regression, we usually want to minimize some suitable loss-function $L$.

\subsection{Algorithms for Classification \& Regression}
\subsubsection{Support Vector Classifiers \& Support Vector Machines (SVMs)}

Support Vector Machines (SVMs)\footnote{A note to the terminology: Most literature refers to the linearly separable case as \emph{Support Vector classifier}, whereas the non--linear high--dimensional enhancement, incorporating kernel methods, are called \emph{Support Vector Machine}.}, originally proposed by \citet{vapnik1963} as linear classification, later extended by \citet{cortes1995} to allow for non--linear classification, are among the most common classification algorithms, especially for high--dimensional datasets. The underlying idea is to find a special \emph{hyperplane} $\hyperplane$, which will act as decision boundary in feature space $\featurespace$.\footnote{This is somewhat similar to \emph{perceptron learning} as devised by \citet{rosenblatt1958}. His algorithm however tries to find a separating hyperplane that minimizes the distance of misclassified points to the decision boundary \citep{hastie2001}.} Among all possible hyperplanes, the optimal SVM decision boundary will be given by the hyperplane which separates the classes \emph{and} maximizes the distance to the closest data points for both classes, therefore called \emph{maximum--margin hyperplane} $\hat \hyperplane$. The data points on the decision boundary are \emph{support vectors}. In any case, $\hat \hyperplane$ is obtained by solving some kind of optimization problem. In the following we will assume that we are dealing with $d$-dimensional training data $\trainingdata$ of length $N$, containing data points of only two classes $\classes = \{ \pm 1 \}$. \\

% @IMAGE: Add image showing data points that are not linearly separable.
% @IMAGE: Add image showing how transformation to higher dimension can make the problem linearly separable

%\paragraph{Support Vector Classifiers}

\emph{If} the data is linearly separable, we fit a linear SVM, which finds the optimal decision boundary

\begin{equation}
\hat \hyperplane = \{ \vec x \where \vec x^T \vec \alpha + \alpha_0 = 0 \},
\end{equation}

with parameters $\vec \alpha, \alpha_0$, obtained by solving the optimization problem

\begin{gather}
\label{eq:svm-linear-hyperplane}
\min_{\vec \alpha, \alpha_0} \frac{1}{2} \| \vec \alpha \|^2 \\
\text{\st} y_i (\vec x_i^T \vec \alpha + \alpha_0) \ge 1 \quad \forall i = 1, \ldots, N,
\end{gather}

where $\alpha_0$ is known as \emph{bias} and $\|\vec \alpha \|^{-1}$ can be identified as the margin, following \citet{hastie2001}. This is a convex optimization problem, so we have standard methods at hand to find the global optimum solution in a computationally efficient way, as outlined by \citet{vanderplas2015}. At this point, the classification problem is solved by $\estimator(\vec x) = \operatorname{sign}(x_i^T \vec \alpha + \alpha_0)$. Obviously, such a hyperplane does not always exist\footnote{In fact, in most real--life problems, $\estimator(\vec x)$ will not be strictly linear in $\vec x$.}, so in practise we have to relax the constraints given by equation \eqref{eq:svm-linear-hyperplane}. For this purpose, we introduce the non--negative \emph{slack variables} $\slack_i$ and slightly modify our optimization problem to \\

% @TODO: Fix equations reference number

\begin{gather}
\label{eq:svm-linear-hyperplane-relaxed}
\min_{\vec \alpha, \alpha_0} \frac{1}{2} \| \vec \alpha \|^2 \\
\text{\st} y_i (\vec x_i^T \vec \alpha + \alpha_0) \ge 1 - \slack_i \quad \forall i = 1, \ldots, N \text{\ and} \\
\text{\st} \sum\limits_{i=1}^{N} \slack_i \le C \in \mathbb{R} \text{\ with\ } \slack_i \ge 0 \quad \forall i = 1, \ldots, N.
\end{gather}

% @TODO: Rephrase
% @TODO: Fix reference number

Essentially, this will allow some points to be misclassified, $\cdots$. This is referred to as \emph{soft--margin SVM}. Note that we have added a free parameter $C$ to our model, that we need to adapt to our specific problem (see section hyperparamater optimization).\\

%\paragraph{Support Vector Machines}

However, even with these relaxed constraints, if the underlying true model is not linear, training a linear model will never yield optimal results, consequently the classification performance will be poor. For this reason \citet{cortes1995} devised an enhancement of the original proposal, which maps the data into a high--dimensional feature space $\hdfeaturespace$ by some (non-linear) mapping function $\hdmapping \colon \mathbb{R}^{d_1} \to \mathbb{R}^{d_2}, d_1 < d_2$, hoping to be able to find a linear decision surface\footnote{Keep in mind that while $\hdhyperplane$ is linear in $\hdfeaturespace$, it will not be linear in $\featurespace$.} in this higher--dimensional representation of the data $\hdmapping(\vec x)$. Yet it is important to realize that $d_2$ can potentially become very large, \eg say that $d_2 \propto 2^{d_1}$, so actually carrying out the transformation $\Phi(\vec x)$ explicitly seems unfeasible due to memory constraints. Fortunately, there is a way to avoid this, known as the \emph{kernel trick}, originally proposed by \citet{aizerman1964}. In their work, \citet{cortes1995} show that in optimization problem \eqref{eq:svm-linear-hyperplane-relaxed}, the data points only contribute to the solution in terms of pairwise dot products $\langle \vec x_i, \vec x_j \rangle$. Therefore, we introduce a \emph{kernel function}\footnote{$\kernel(\vec x, \vec x')$ can be seen as similarity measure for $\vec x, \vec x'$.} or \emph{kernel}

\begin{equation}
\label{eq:kernel-function}
\kernel \colon \observations \times \observations \to \mathbb{R},\, (\vec x, \vec x') \mapsto \langle \phi(\vec x), \phi(\vec x') \rangle,
\end{equation}\\

% Rephrase

where $\langle \cdot, \cdot \rangle$ denotes the dot product, and $\vec x'$ is generally the reference centre, but in the process of training the SVM, $\vec x'$ will be the unlabelled data point under consideration. In light of equation \eqref{eq:kernel-function}, $\kernel(\vec x, \vec x')$ can be expressed as dot product between the images of $\vec x$ \resp $\vec x'$ under $\phi$. We are not interested in any direct form of $\phi$, but its existence is assured by Mercer's theorem, which --- loosely speaking --- states that $\phi$ exists if and only if $\kernel(\vec x, \vec x')$ is continuous, symmetric \emph{and} positive semi--definite \citep{mercer1909}, otherwise $\kernel(\vec x, \vec x')$ will not resemble a dot product. Finally, we see that we can have our SVM operate in $\hdfeaturespace$ instead of $\featurespace$ in a very efficient manner just by substituting all occurrences of $\langle \vec x_i, \vec x_j \rangle$ in the corresponding dual formulation of equation \eqref{eq:svm-linear-hyperplane-relaxed} with $\kernel({\vec x_i, \vec x_j})$ for all $i,j$. \\
 
A suitable choice for $\kernel$ does ultimately depend on the problem, ideally incorporating domain--knowledge, but a very common choice is the Gaussian \emph{radial basis function} (RBF)\footnote{All radial basis functions $f$ must satisfy $f(\vec x) = f(\|\vec x\|) \; \forall \vec x$.} kernel

\begin{equation}
\label{eq:rbf}
\kernel_{\text{RBF}}(\vec x, \vec x') = \euler^{-\gamma \| \vec x - \vec x' \|^2},
\end{equation}

with $\gamma = -\frac{1}{2 \sigma^2}$.\\

% RBF -> Hilbert space

Technically, SVMs are restricted to binary classification problems, that is data sets with only two distinct classes, but in practise every multi--class classification problem can be transformed into multiple binary classification problems, \eg by training multiple SVMs for either one class against all classes (\emph{one--vs.--all}), or every class against every other class (\emph{one--vs.--one}) \citep{knerr1990}. In the end, for $N$ classes, we would actually train $N \cdot \frac{N-1}{2}$ classifiers when using the \emph{one--vs.--all} approach.

% Scale data
% List common choices for kernels

% @TODO: Name advantages and disadvantages

\subsubsection{Decision Trees \& Random Forests}

Decision trees\footnote{There are a variety of slightly different algorithms for tree construction (ID3, C4.5, \etc), splitting, pre-- and post--processing described in the literature. Unless otherwise noted, we will talk about \emph{Classification and Regression Trees} (CART) as described by \citet{breiman1984}.} \citep{breiman1984} are presumably the most intuitive way to subdivide data into different classes, and consequently most likely the way humans would tackle classification problems. The basic idea is to iteratively split the training set during the training process into exactly two subsets by asking a sequence of binary questions, until some predefined stop criteria is satisfied. This procedure, called \emph{recursive partitioning}, gives rise to a \emph{decision tree} consisting of inner nodes with exactly two children, representing split decisions, and leaf nodes representing labels.\footnote{This very structure is called a \emph{full binary tree} in  theoretical computer science and its properties are well characterized \citep{knuth1981}. This of course also means that efficient memory representations exist for such an entity.} With every split the tree grows deeper, and the remaining feature set becomes smaller. In the end, given some testing data $\trainingdata$, classification is performed simply by walking down the tree, and the corresponding label $\hat y_i$ is given by the respective leaf node. Due to the nature of binary questions, all branches are ``mutually distinct and exhaustive'' \citep{duda2001}, so exactly one branch will be followed, which means that $\hat y_i$ is unambiguous. One might wonder why binary splits are preferred over multi--way splits. The main reason for this is that, unless we are talking about a tremendous amount of data, multi--way splits will rapidly thin out our data at each level, because of the broad fragmentation at each decision node \citep{hastie2001}. % Still we can achieve that with a series of binary splits...
\\

% @TODO: Computational considerations: Duda, p. 406, 407
% @TODO: Explain at what feature the split should happen

% @IMAGE: Add image depicting decision tree

The effectiveness of the above approach will ultimately depend on the choice of split criteria. A good split would ideally single out one of the classes, producing a perfectly pure node that only consists of members of the respective class. There are a variety of metrics to access the purity or impurity of every node in a classification tree. Let $p_i = \frac{1}{N} \sum\limits_{\vec x_i \in c_i}^N 1$ be the relative frequency of the $i^{\text{th}}$ class $c_i \in \classes$ in the remaining set of $N$ observations. We define the \emph{Gini impurity} or \emph{Gini coefficient} for $k = |\classes|$ classes as

\begin{equation}
\label{eq:gini-impurity}
G_{\text{I}} = \sum\limits_{i=1}^k p_i (1 - p_i),
\end{equation}

following \citet{astroML,hastie2001,ripley2007}. Furthermore we define the \emph{entropy} or \emph{deviance} of a set as

\begin{equation}
\label{eq:entropy-impurity}
E_{\text{I}} = - \sum\limits_{i=1}^k p_i \ln(p_i).
\end{equation}

Based on equation \eqref{eq:entropy-impurity}, we can introduce the \emph{information gain} or \emph{Kullback--Leibler divergence} \citep{kullback1951} for a binary split

\begin{equation}
\label{eq:information-gain}
I = E_{\text{I}} - \frac{1}{N} \big(N_a E_{\text{I}}^{(a)} + N_b E_{\text{I}}^{(b)}\big),
\end{equation}

where $N_a, N_b$ denote the number of data points above and below the split threshold $S$. Thus, when growing the tree, we perform the split that maximizes the information gain, \ie solving $\argmax_S I$.\\

% @TODO: Add suitable metric for regression trees

In order to avoid overfitting and retain generalization, we have to set an upper bound for our model complexity by limiting the depth $h$ of our tree. Different strategies have been proposed to achieve this:

\begin{enumerate}
\item \label{itm:constant-metric} Stop growing sub--trees when further splits do not affect the desired impurity metric by more than some constant $\delta$. This is good choice in case the complexity of the features differs substantially among the data \citep{duda2001}, leading to varying position of the leafs in the tree\footnote{At this point the tree becomes \emph{unbalanced}.}. A major drawback is, of course, the free parameter $\delta$ which has to be optimized.
\item \label{itm:constant-data-points} Simply stop growing when $N \le N_c$ with some preset number of data points $N_c$. Once again, this adds another parameter to the model.
\item \label{itm:validation} \emph{Validation}: Use cross--validation or some other method to compare training performance with testing performance. This could become computationally expensive, especially if it has to be performed at every node.
\item \label{itm:pruning} \emph{Pruning}: % @TODO: Elaborate on pruning (Duda, p. 403, 403; Hastie, p. 208)...
\item A combination of the illustrated strategies.
\end{enumerate}

% One way to avoid this is bagging. ...
% Another approach is random forests \citep{...}...
% Write about stop criteria, optimal depth, overfitting
% Write about regression trees
% Random Forests essentially limits the number of features on which the tree is constructed


% @TODO: Name advantages and disadvantages for Decision trees (see e.g. http://scikit-learn.org/stable/modules/tree.html)

%%%
% DECISION TREES
%%%

% Top-down induction of decision trees 
% Optimal depth of the tree needs to be determined
% Normal decision trees are prone to overfitting
% Very intuitive, very easy to interpret
% Different split criteria: Gini coefficient, entropy (information gain)
% Returns feature importance
% Pruning
% Bagging and Random Forests

%%%
% RANDOM FORESTS
%%%

Another way to restrict the model complexity is known as \emph{Random Forests} \citep{breiman2001}, which is an example of \emph{ensemble learning}.

% Decision forests
% Ensemble Learning
% Random Forests: Features on which to generate the tree on are randomly drawn
% Random Forests: no axis alignment

% @TODO: Name advantages and disadvantages for Random forests (see e.g. http://www.stat.berkeley.edu/~breiman/RandomForests/cc_home.htm)

%\section{Dimensionality reduction on high-dimensional datasets}
\label{sec:dimensionality-reduction}

Often we are confronted with high--dimensional datasets that presumably contain a lot of redundancy. This means that the interesting data can be embedded in a lower--dimensional manifold (\emph{almost everywhere}). Therefore, before tackling the actual problem to be solved, often a so called dimensionality reduction is performed on the dataset. That is trying to find a lower--dimensional shadow of the data, which still contains most of the (relevant) information. Reasons to do this are manifold, but the dominant motivation is certainly to cut down on computation time and memory usage, which is often the limiting factor --- even on sophisticated high--performance platforms. Also, some algorithms used in machine learning suffer from the \emph{curse of dimensionality}. Another reason is that visualization is generally easier in a low--dimensional subspace.\\

In the following we give a brief description of two standard methods that are later used to do dimensionality reduction on our dataset.

\subsection{Principal component analysis (PCA)}
\label{sec:theory:pca}

Principal component analysis\footnote{Sometimes also referred to as \emph{Karhunen-Loève transformation}.} (PCA) is a standard method to reduce the number of variables in a dataset by trying to find linear combinations of features that maximize the total variance in the data, allowing for minimal loss of information. Since it does not care of the labels, PCA is considered as unsupervised learning. In terms of vector spaces, PCA performs a linear transformation of the data from a $d$--dimensional space to a $k$--dimensional subspace (in any case $k < d$), where the data is most spread out. This subspace can be identified as the span of the $k$ orthogonal dominant components --- therefore called \emph{principal components} ---, defining a new coordinate system. Because of the orthogonality, we find that the new features are linearly uncorrelated. Evidently, $k$ is a free parameter, that we need to choose small enough to obtain a significant data reduction, but not so small that important information is lost. In general, we say that a subspace describes the data well, if the eigenvalues of the scatter matrix $S$ (that we computed during the PCA) are similar of magnitude.\\

There are essentially two similar ways to perform the PCA, one using the scatter matrix, the other one using the covariance matrix. We provide an algorithmic description using the former \citep{duda2001}.

\subsubsection{Algorithmic description of PCA}
\label{par:pca-algorithm}
\begin{enumerate}
\item \footnote{Since PCA is sensitive to the scaling of the data, often also a normalization is performed in advance (e.g. z--score). However, we do not consider this as part of the algorithm itself.}Find the $d$--dimensional mean vector
\begin{equation}
\label{eq:mean-vector}
\vec \mu = \frac{1}{n} \sum_{i = 1}^n \vec x_i
\end{equation}
of all features $\vec x_i$ in the $n \times d$--dimensional dataset.
\item Find the scatter matrix\footnote{Note that because $S$ is symmetric and $S_{ij} \in \mathbb{R} \, \forall i,j$, the eigenvectors are orthogonal.}
\begin{equation}
S = \sum_{i=1}^n (\vec x_i - \vec \mu) \, (\vec x_i - \vec \mu)^T
\end{equation}
and calculate eigenvectors and the respective eigenvalues.
\item Take the $k$ eigenvectors with the $k$ largest eigenvalues and construct a $d \times k$--dimensional matrix $P$.
\item Project the data into the new subspace using the linear transformation:
\begin{equation}
\vec y = P^T \cdot \vec x
\end{equation}
\end{enumerate}

PCA is one of the simpler methods for dimensionality reduction based on eigenvectors. It is very straightforward to perform, but has some drawbacks, as well. For our task (classification), PCA is not the first choice, because it is not designed to maintain class separability.

\subsection{Linear Discriminant Analysis (LDA)}
\label{sec:theory:lda}

Linear Discriminant Analysis (LDA) is another way of performing dimensionality reduction, this time, in contrast to PCA, a supervised learning algorithm that takes the labels into account\footnote{Note that LDA is also commonly used for classification.}. Again we are trying to project into a lower--dimensional subspace, maximizing the scatter in feature space, but \emph{additionally} minimizing the in--class scatter, consequently keeping the class--discriminatory information. Ideally, each class degenerates to a cluster far away from other clusters in the new feature space with respect to an arbitrary metric.

\subsubsection{Algorithmic description of LDA}
\begin{enumerate}
\item Find the $d$--dimensional mean vectors $\mu_i$ (see equation \eqref{eq:mean-vector}) for each class
\item Find the within--class scatter matrix
\begin{equation}
S_W = \sum_{i = 1}^m S_i,
\end{equation}
\begin{equation}
S_i = \sum_{\vec x \in D_i}^n (\vec x - \vec \mu_i) \, (\vec x - \vec \mu_i)^T,
\end{equation}
where $m$ is the number of classes and $S_i$ is the scatter matrix of the $i$--th class.
\item Find the between--class scatter matrix
\begin{equation}
S_B = \sum_{i = 1}^m N_i (\vec \mu_i - \vec \mu) \, (\vec \mu_i - \vec \mu)^T,
\end{equation}
where $\vec \mu$ denotes the mean vector of the whole dataset.
\item Find (generalized) eigenvectors and the respective eigenvalues for $S := S_W^{-1} S_B$. These will be used as linear discriminants.
\item Take the $k$ eigenvectors with the $k$ largest eigenvalues and construct a $d \times k$--dimensional matrix $P$.
\item Project the data into the new subspace using the linear transformation:
\begin{equation}
\vec y = P^T \cdot \vec x
\end{equation}
\end{enumerate}

In practise, we see that very often a combination of both procedures is performed, e.g. PCA followed by LDA. This is because LDA is computationally more expensive than PCA.
